	
%%%%%%%%%%%%%%%%%%%%%%%%%%%%%%%%%%%%%%%%%%%%%%%%%%%%%%%%%%%%%%%%%%%%%%%%%%%%%%%%%%%%%%%%%%%%%%%%%%%%
%%%%%%%%%%%%%%%%%%%%%%%%%%%%%%%%%%%%%%%%%%%%%%%%%%%%%%%%%%%%%%%%%%%%%%%%%%%%%%%%%%%%%%%%%%%%%%%%%%%%
%%%%%%%%%%%%%%%%%%%%%%%%%%%%%%%%%%%%%%%%%%%%%%%%%%%%%%%%%%%%%%%%%%%%%%%%%%%%%%%%%%%%%%%%%%%%%%%%%%%%
%%%%%%%%%%%%%%%%%%%%%%%%%%%%%%%%%%%%%%%%%%%%%%%%%%%%%%%%%%%%%%%%%%%%%%%%%%%%%%%%%%%%%%%%%%%%%%%%%%%%
%%%%%%%%%%%%%%%%%%%%%%%%%%%%%%%%%%%%%%%%%%%%%%%%%%%%%%%%%%%%%%%%%%%%%%%%%%%%%%%%%%%%%%%%%%%%%%%%%%%%
%%%%%%%%%%%%%%%%%%%%%%%%%%%%%%%%%%%%%%%%%%%%%%%%%%%%%%%%%%%%%%%%%%%%%%%%%%%%%%%%%%%%%%%%%%%%%%%%%%%%
\clearpage
\section{Depth}

\subsection{Computing and Design Alternatives}

		\citeSK{marks1997design}
		\begin{quote}
		\small
		An approach to 
		1) compute a large number of computationally-expensive graphics,
		2) summarize subjective qualities of each of these graphics as an (objective) output vector, and
		3) present these graphics through some \emph{arrangement method} (Fig. \ref{fig:marks1997design}).
		The so-called \emph{dispersion method} picks input vectors so that output vectors (and thus
		output graphics) are well-distributed. Output vectors have \emph{distance metric} which is used
		to summarize subjectively relevant qualities of output graphics. 
		The method presented was applied to three different cases:
		image rendering using raytracing,
		volume rendering, and
		motion control.
		
		\begin{figure}[h!]
		\centering
		\begin{tikzpicture}
		  [
		  scale=1.0,
		  ]
		  
		  %\draw [help lines] (0,0) grid (10,4);
			
		% 	\tikz\draw (0,0) ellipse (20pt and 28pt);
		% 	\tikz\draw (4,0) ellipse (20pt and 28pt);
		% 	\tikz\draw (8,0) ellipse (20pt and 28pt);
			%\draw (0,0) node[ellipse, minimum height=2cm,minimum width=3cm,draw] {Ellipse};
			%\path (0,0)  node(a) [ellipse,rotate=0,draw,fill]    {An ellipse}
			
			\draw (0,0) circle [x radius=0.5, y radius=1.5];
			\node (IV) at (0,2) {\small set of input vectors};
		
			\draw (4,0) circle [x radius=0.5, y radius=1.5];
			\node (OG) at (4,2) {\small set of output graphics};
			
			\draw (8,0) circle [x radius=0.5, y radius=1.5];
			\node (OV) at (8,2) {\small set of output vectors};
			
			\draw[->, >=triangle 45] (1,0) -- (3,0) node[near start, above] {\small \ \ \ \ mapping 1:1};
		
			\draw[->, >=triangle 45] (5,0) -- (7,0) node[near start, above] {\small \ \ \ \ ``summary'' 1:1};
			%\node[near start, below] at (7,0) {\small mapping};
			
			\node (OVmetric) at (8, -2) {\small has distance metric $d$};
			
			\node (OG1) at (4, -2) {\small has arrangement};
			\node (OG1) at (4, -2.5) {\small (makes use of $d$)};
			
			\node (OVmetric) at (0, -2) {\small has dispersion method};
			
			\draw (current bounding box.north east) rectangle (current bounding box.south west);
				
		\end{tikzpicture}
		
		\caption{A graphical summary of most important concepts in Design Galleries by Marks et al \cite{marks1997design}.}
		\label{fig:marks1997design}
		\end{figure}
		
		\end{quote}



\citeSK{marks1997designbrowsers}
		\begin{quote}
		\small
		The authors describe and showcase several browsers (i.e. browsing modalities) 
		for Design Galleries \cite{marks1997design},
		both 2D and 3D (Fig. \ref{fig:marks1997designbrowsers}). The 3D version
		is in some aspects more usable than the 2D version (for example, providing some
		insights that the 2D version cannot provide).
		\begin{figure}[htb]
		\begin{center}
		\includegraphics[width=3in]{figures/marks1997designbrowsers.jpg}
		\caption{3D browser for Design Galleries \cite{marks1997designbrowsers}.}
		\label{fig:marks1997designbrowsers}
		\end{center}
		\end{figure}				
		\end{quote}

\citeSK{roberts2000multiple}
		\begin{quote}
		\small
		Describes {\em multiplicity} in visualization, 
		where data are shown in separate windows (Fig. \ref{fig:roberts2000multiple}).
		The authors lists five reasons for using multiple views: 
		correct dissemination,
		scientific exploration,
		alternative viewpoints,
		realization comparison, and
		collaboration.
		\begin{figure}[htb]
		\begin{center}
		\includegraphics[width=4in]{figures/roberts2000multiple.jpg}
		\caption{Multiplicity in visualization \cite{roberts2000multiple}.}
		\label{fig:roberts2000multiple}
		\end{center}
		\end{figure}
		\end{quote}
		
\citeSK{jankun2000spreadsheet}
		\begin{quote}
		\small
		The authors present a spreadsheet-like interface for exploring and displaying multidimensional 
		parameter spaces and their associated visualizations (Fig. \ref{fig:jankun2000spreadsheet}). 
		The visualizations themselves (i.e. output graphics) are organized in a two-dimensional tabular form.
		Axes (X, Y) show values for two parameters, at a time, while the rest of the parameters take on
		their default values.
		\begin{figure}[htb]
		\begin{center}
		\includegraphics[width=4in]{figures/jankun2000spreadsheet.jpg}
		\caption{Spreadsheet-like interface \cite{jankun2000spreadsheet}.}
		\label{fig:jankun2000spreadsheet}
		\end{center}
		\end{figure}
		\end{quote}

%\citeSK{derthick2001enhancing} \\


\citeSK{terry2002recognizing}
		\begin{quote}
		\small
		Analyzing three case studies related to image manipulation, 
		%(image toning, user interface design, and pen drawing coloring), 
		and arguing that computer systems need to better support creative processes, 
		authors present a set of general user interface design guidelines which would allow the user to 
		1) \emph{experiment} with commands, 
		2) \emph{generate and explore variations} of document components, and
		3) \emph{evaluate document states}.
		\end{quote}

\citeSK{quigley2002semi}
		\begin{quote}
		\small
% 		Generally it is very difficult to capture subjective aesthetic and functional criteria as
% 		an objective (mathematical or computational) function. 
		This paper presents a mixed-initiative
		approach to antenna design, where computer is used to sample a space of possible antenna designs, and 
		visualize these samples using a variant of Design Galleries (see Fig. \ref{fig:quigley2002semi}
		and Marks et al \cite{marks1997design}); 
		human input is then used to select samples according to some subjective criteria.
		\begin{figure}[htb]
		\begin{center}
		\includegraphics[width=4in]{figures/quigley2002semi.jpg}
		\caption{Antenna designs \cite{quigley2002semi}.}
		\label{fig:quigley2002semi}
		\end{center}
		\end{figure}
		\end{quote}
		
%\citeSK{stolte2002polaris}  \\

\citeSK{stump2003design}
		\begin{quote}
		\small
		Demonstrates a graphical user interface for visualizing complex design spaces,
		containing rich sets of ``good'' designs. Supports generation, discovery,
		comparison, selection and recombination of designs. The interface makes use of multi-dimensional
		visualization techniques (glyph plots, parallel coordinates, linked views,
		brushing, histograms, Pareto frontier) and interaction techniques (specifying lower and upper
		bounds on the design space variables, assigning variables to glyph axes and parallel
		coordinate plots, and brushing variables dynamically). An example of how to
		use the interface is given (satellite design at Lockheed Martin).
		\begin{figure}[htb]
			\begin{center}
			\includegraphics[width=0.99\textwidth]{figures/stump2003design.jpg}
			%
			\caption{Customizable glyphs for displaying multivariate information on designs (left), 
			and designs on the Pareto frontier (right, dark spheres). 
			\citeauthor{stump2003design}: 
			\citetitle{stump2003design} 
			\cite{stump2003design}.}
			%
			\label{fig:stump2003design}
			\end{center}
		\end{figure}
		\end{quote}


\citeSK{terry2004variation}
		\begin{quote}
		\small
		The paper describes a general approach ({\em Parallel Paths}) 
		to implementing user-initiated development of alternatives/variations for any given document.
		It also presents a way to visualize and to work with a number of alternatives on the screen
		({\em Parallel Pies}).
		Possible problems with the approach:\\
		- no word on storage requirements (space complexity) when copying the current document
		  into a new alternative. For example, images can be very large.\\
		- no word on copying just parts (of the document) into an alternative (and not only the whole
		  document).\\
		- Parallel Pies works well for several (2, 3, \dots) variations, but what if there are many more
		  variations (for example, forty of them)? Slices in the pie probably become so 
		  narrow as to make proper viewing of an alternative impossible.\\
		- commands work on just one or all (1 or N) variations---what if we want to work with a subset
		  of variations, for example half of them (N/2)?\\
		- there is no need to copy the whole history when making an alternative; just adding/linking a new 
		  node in the history tree/graph would probably be enough.
		\end{quote}

\citeSK{terry2005phdthesis}
		\begin{quote}
		\small
		This thesis suggests easier ways 
		to develop and explore entire {\em sets of alternative design
		solutions}, i.e. a number of alternatives in parallel.
		The author presents two software prototypes, 
		{\em Side Views} (which applies a type of a tooltip mechanism, only showing the
		effect of a command, and/or a series of previews across some range of values,
		for each parameter in the command), and
		{\em Parallel Pies} (a visualization/interaction modality which allows a user
		to both visualize multiple alternatives on the same screen, and easily create
		new alternatives). To evaluate the impact of these tools on problem solving,
		the author conducts three studies (two controlled, and one think-aloud study).
		Experimental results related to Parallel Pies indicate both negative effects
		(use of Parallel Pies can lead to lower-quality results) and positive effects 
		(users explore problems more broadly). 
		Experimental results related to Side Views indicate 
		either no effects (i.e. no change in user performance) or positive effects 
		(easier discovery of unexpected/unplanned-for but still viable solutions).
		\end{quote}

%\citeSK{bavoil2005vistrails}  \\

\citeSK{tohidi2006getting} 
		\begin{quote}
		\small
		In this experimental study, authors used low-cost paper prototyping of three
		different human-computer interfaces (Fig. \ref{fig:tohidi2006getting}) 
		for a "house climate control system" (HCCS).
		The participants (N=48) were then exposed to:\\
		- one single interface (three groups of 12, one group for each interface), or\\
		- all three interfaces (one group of 12).\\
		Experimental results (between-subjects) support the following claims:\\
		- single-design users were more reluctant to criticize the design presented to them.\\
		- single-design users gave significantly higher rankings to a design.\\
		- usability testing is a means to identify problems, not provide solutions.
		\begin{figure}[htb]
		\begin{center}
		\includegraphics[width=0.99\textwidth]{figures/tohidi2006getting.jpg}
		\caption{Three different GUI mockups \cite{tohidi2006getting}.}
		\label{fig:tohidi2006getting}
		\end{center}
		\end{figure}
		\end{quote}

% \citeSK{jennings2007design}
% 		\begin{quote}
% 		\small
% 		\end{quote}


\citeSK{roberts2007state}
		\begin{quote}
		\small
		The author argues that while much work has been done in the 
		field of {\em coordinate and multiple views} (Fig. \ref{fig:roberts2007state}),
		lots of work still has to be done, and 
		that the emphasis of CMV is slowly shifting towards visual analytics (VA).
		The author points out areas which need work, and several promising areas of future work.
		\begin{figure}[htb]
		\begin{center}
		\includegraphics[width=0.6\textwidth]{figures/roberts2007state.jpg}
		\caption{Multiple coordinated views \cite{roberts2007state}.}
		\label{fig:roberts2007state}
		\end{center}
		\end{figure}
		\end{quote}

\citeSK{hartmann2008design}
		\begin{quote}
		\small
		This paper describes editing and execution of multiple program variations 
		{\em in parallel} (Fig. \ref{fig:hartmann2008design}).
		The prototype they showcased ({\em Juxtapose}) is able both to support variations of source code(s),
		and the parallel execution of runtime codes corresponding to these source codes.
		\begin{figure}[htb]
		\begin{center}
		\includegraphics[width=0.99\textwidth]{figures/hartmann2008design.jpg}
		\caption{Parallel code alternatives (left) and 
		parallel executed codes (right), 
		in \citeauthor{hartmann2008design}: 
		\citetitle{hartmann2008design} 
		\cite{hartmann2008design}.}
		\label{fig:hartmann2008design}
		\end{center}
		\end{figure}
		\end{quote}



\citeSK{lunzer2008subjunctive}
		\begin{quote}
		\small
		In this paper, authors propose a novel {\em subjunctive} interface 
		and report experimental results after evaluating the same.
		In subjunctive interfaces, GUI elements can assume multiple states simultaneously 
		(Fig. \ref{fig:lunzer2008subjunctive}).
		Experimental results indicate higher user satisfaction,
		and shorter (up to 27\%) task completion times.
		\begin{figure}[htb]
		\begin{center}
		\includegraphics[width=2in]{figures/lunzer2008subjunctive.jpg}
		\caption{Subjunctive interface as suggested by Lunzer and Hornb{\ae}k \cite{lunzer2008subjunctive}.}
		\label{fig:lunzer2008subjunctive}
		\end{center}
		\end{figure}
		\end{quote}

%\citeSK{heer2008graphical}  \\

\citeSK{herring2009getting}
		\begin{quote}
		\small
		The paper describes a field study of how expert designers (N=11) in web,
		product and graphic design use {\em examples} in their work.
		The method used was semi-structured interviews (done in designers' own offices,
		in order to situate them in their usual working environment), with 16 questions focusing
		on benefits and limitations of examples; example retrieval, storage and 
		dissemination techniques. The recommendations by authors are: 
		1) augment search to prevent fixation on a particular design,
		2) improve capture and visualization of search results, 
		3) integrate physical and digital sources of examples (e.g. by taking pictures),
		4) help designers recall why they stored examples, and
		5) encourage contributions of personal examples.		
		\begin{figure}[htb]
		\begin{center}
		\includegraphics[width=3in]{figures/herring2009getting.jpg}
		\caption{Examples in a design studio; 
		in \citeauthor{herring2009getting}: \citetitle{herring2009getting} \cite{herring2009getting}.}
		\label{fig:herring2009getting}
		\end{center}
		\end{figure}
		\end{quote}

\citeSK{schipper2009visual}
		\begin{quote}
		\small
		The paper proposes several approaches to visual comparison of diagrams used in graphical modeling
		of complex computer systems (Fig. \ref{fig:schipper2009visual}). 
		The authors claim that it is possible, using these approaches, to compare even very large models, 
		without missing critical details. The means used: 
		structural comparisons,
		colored side-by-side comparisons,
		automatic layout,
		navigation and 
		folding.
		\begin{figure}[htb]
		\begin{center}
		\includegraphics[width=3in]{figures/schipper2009visual.jpg}
		\caption{Visual comparison of models in Schipper et al \cite{schipper2009visual}.}
		\label{fig:schipper2009visual}
		\end{center}
		\end{figure}				
		\end{quote}

\citeSK{sheikholeslami2009mscthesis} 
		\begin{quote}
		\small
		This thesis describes {\em hysterical spaces} based on 
		Cartesian product of sets of parameter values altered by the user.
		The author goes on to develop a mathematical formalism capturing the notion
		of such spaces, as well as to suggest ways to interact with their elements. 
		\end{quote}

\citeSK{busking2010dynamic}
		\begin{quote}
		\small
		The paper proposes a visualization application (Fig. \ref{fig:busking2010dynamic}) 
		for visual exploration of
		{\em shape spaces} (composed of points representing a 2D or 3D surface of objects), and 
		{\em shape models} (statistical distributions of objects in shape spaces). Main
		contributions of the article:\\
		- strongly-linked multi-view approach to shape exploration,\\
		- shape evolution view,\\
		- high performance based on GPU computing.
		\begin{figure}[htb]
		\begin{center}
		\includegraphics[width=2.5in]{figures/busking2010dynamic.jpg}
		\caption{Exploration of shape spaces in Busking et al \cite{busking2010dynamic}.}
		\label{fig:busking2010dynamic}
		\end{center}
		\end{figure}				
		\end{quote}
				
\citeSK{dow2010parallel}
		\begin{quote}
		\small
		In the context of Web banner advertisement design, this paper investigates what is 
		better under time constraints --- parallel or serial prototyping?
		Experimental results show that parallel prototyping outperformed serial prototyping
		in all the measures used in the experiment. The conclusions of the study were that
		parallel prototyping (i.e. simultaneous creation of multiple prototypes) 
		gives better design results and more ideas, and that users
		who applied it react more positively to design critique, as compared to serial
		prototyping (which is basically equivalent to the iteration of one single design).
		\end{quote}

\citeSK{lee2010designing}
		\begin{quote}
		\small
		This paper examines whether human-computer interfaces offering {\em interactive example galleries} 
		help users to find inspirational examples which can then be modified to create new designs
		({\em design by example modification}). To answer this question the authors developed 
		a prototype web page editor ({\em Adaptive Ideas} --- see Fig. \ref{fig:lee2010designing}) 
		and conducted
		three different usability experiments whose results show that a body of examples
		can improve design work.
		\begin{figure}[htb]
		\begin{center}
		\includegraphics[width=3.5in]{figures/lee2010designing.jpg}
		\caption{Lee et al \cite{lee2010designing} --- {\em Adaptive Ideas} prototype for interactive
		example galleries.}
		\label{fig:lee2010designing}
		\end{center}
		\end{figure}				
		\end{quote}


\citeSK{warth2011worlds} 
		\begin{quote}
		\small
		The authors present an approach to isolating and controling side effects in programming languages,
		by introducing {\em worlds}. A world acts as a sort of a sandbox containing all side effects 
		(i.e. changes to some parts or all of program state), thus enabling easy experimentation. 
		A new world can be {\em sprouted} from its originating world; side effects then take place in this
		child world and stay contained within it, unless we explicitly propagate them to its parent world by 
		executing a commit operation.
		\end{quote}
		
\citeSK{chen2012phdthesis} 
%\citeSKextended{chen2012phdthesis}{Chapter 6: Design of the Formalism}  \\
		\begin{quote}
		\small
		This thesis presents a formalism for representing alternatives.
		The first part of the thesis gives motivation (stemming from the real-world usage of CZSaw,
		a visual analytics [VA] system), survey of several other VA systems,
		prior work related to representing alternatives (including but not limited to: 
		sofware configuration management, directed acyclic graphs, typed feature structures, AND/XOR trees),
		and the methodology used to evaluate the formalism.
		The second part of the thesis develops and present the basic design of the formalism 
		(includes definitions of variations, variation heads, unification mechanisms, and variation spaces).
		Of special interest is Chapter 6, which treats in depth the notions of 
		variations (vs. more general {\em alternatives})
		variation properties,
		unbounded variation properties (one which has no information, i.e. its value is unknown),
		variation spaces,
		operations over variation heads (unification and subsumption),
		unification of variation spaces, and
		indices (variations within a variation space are ordered by their indices).		
		\end{quote}


\clearpage		
\subsection{Design Alternatives ``in the Wild''}	


%\clearpage
\citeSK{davinci1485sforza}
		\begin{quote}
		\small
		Study of an equestrian movement by da Vinci, intended to show Duke Francesco Sforza mounted on horseback.
		The drawing features several regions of overlaid alternative details (with different positioning of parts),
		thus allowing the viewer to imagine a number of different final outcomes.
		\begin{figure}[htb]
		\begin{center}
		\includegraphics[width=0.99\textwidth]{figures/davinci1485sforza1a.jpg}
		\caption{
		%add
		\citeauthor{davinci1485sforza}: 
		\citetitle{davinci1485sforza} 
		\cite{davinci1485sforza}.
		Image adapted from: 
		FriendsOfArt.com (\url{http://www.friendsofart.net/static/images/art1/leonardo-da-vinci-study-for-the-sforza-monument.jpg}).}		
		\label{fig:davinci1485sforza}
		\end{center}
		\end{figure}		
		\end{quote}

\clearpage
\citeSK{davinci1487vitruvian}
		\begin{quote}
		\small
		Drawing of a male figure by da Vinci, based on human body proportions given in Book III of Vitruvius' {\em Ten Books on Architecture}. 
		The drawing places two superimposed figures on a single sheet of paper,
		thus depicting two alternative poses at the same time.
		\begin{figure}[htb]
		\begin{center}
		\includegraphics[width=0.65\textwidth]{figures/davinci1487vitruvian.jpg}
		\caption{
		%add
		\citeauthor{davinci1487vitruvian}: 
		\citetitle{davinci1487vitruvian} 
		\cite{davinci1487vitruvian}.
		Source: Wikimedia Commons.}		
		\label{fig:davinci1487vitruvian}
		\end{center}
		\end{figure}		
		\end{quote}

				
\clearpage
\citeSK{siza1977malagueira}
		\begin{quote}
		\small
		This example of working with design alternatives 
		shows multiple sketches done by \'Alvaro Joaquim de Melo Siza Vieira, 
		a Portuguese architect, winner of the Pritzker Prize in 1992.
		The project in question was called Bairro Quinta da Malagueira, 
		a 27-hectare residential development project for around 12,000 people,
		to the west of the city of \'Evora in Portugal.
		\begin{figure}[htb]
		\begin{center}
		\includegraphics[width=0.85\textwidth]{figures/siza1977malagueira.jpg}
		\caption{
		Sketches and floorplans of house and patio variations by \'Alvaro Siza.
		Left: 3D sketches. % (one-storey in upper half, two-storey in lower half). 
		Right: Type 1: front patio toward the street. Type 2: patio at the back. 
		\citeauthor{siza1977malagueira}: 
		\citetitle{siza1977malagueira} 
		\cite{siza1977malagueira}.
		Sketches and floorplans adapted from 
		\citeauthor{fleck1995siza}: 
		\citetitle{fleck1995siza} 
		\cite{fleck1995siza}.}		
		\label{fig:siza1977malagueira}
		\end{center}
		\end{figure}
		
		\begin{figure}[htb]
		\begin{center}
		\includegraphics[width=0.7\textwidth]{figures/siza1977malagueira1.jpg}
		\caption{
		Completed project: \'Alvaro Siza's Quinta de Malagueira.
		Photos adapted from 
		\citeauthor{fleck1995siza}: 
		\citetitle{fleck1995siza} 
		\cite{fleck1995siza}.
		}
		\label{fig:siza1977malagueira1}
		\end{center}
		\end{figure}		
				
		\end{quote}


\clearpage
\citeSK{starck1990juicysalif}

		\begin{quote}
		\small
		
		This is a lemon/citrus squeezer for which the ideation was 
		famously done in one single day, in a Corsican restaurant.
		Figure \ref{fig:starck1990juicysalif2} shows the ideation sketched by Starck on a restaurant 
		napkin/table mat, and Fig. \ref{fig:starck1990juicysalif} shows
		the final manufactured artifact.
		
		\begin{figure}[htb]
		\begin{center}
		\includegraphics[width=0.99\textwidth]{figures/starck1990juicysalif2.jpg}
		\caption{Philippe Starck's ideation for Juicy Salif citrus squeezer, done on a 
		restaurant table mat; the sketches proceed in counterclockwise manner, starting
		from the lower right corner \cite{lloyd2003philippe}.
		Adapted from 
		\url{http://alysapar22.wordpress.com/2010/06/29/expreienc-juicy-salif/}.}
		\label{fig:starck1990juicysalif2}
		\end{center}
		\end{figure}		
		
		\begin{figure}[htb]
		\begin{center}
		\includegraphics[width=0.5\textwidth]{figures/starck1990juicysalif.jpg}
		\caption{Completed artifact: Philippe Starck's {\em Juicy Salif} citrus squeezer.}
		\label{fig:starck1990juicysalif}
		\end{center}
		\end{figure}		
		
		\end{quote}

		
\clearpage
\citeSK{piano1998tjibaou}
		\begin{quote}
		\small
		A cultural center completed in 1998, for which the architect created a number of preliminary sketches.
		
		\begin{figure}[htb]
		\begin{center}
		\includegraphics[width=0.99\textwidth]{figures/piano1998tjibaou1b.jpg}
		\caption{
		Sketches for
		\citeauthor{piano1998tjibaou}: 
		\citetitle{piano1998tjibaou} 
		\cite{piano1998tjibaou}.		
		Adapted from 
		\citeauthor{jodidio2005renzo}: 
		\citetitle{jodidio2005renzo} 
		\cite{jodidio2005renzo}.
		}		
		\label{fig:piano1998tjibaou1}
		\end{center}
		\end{figure}		
		
		\begin{figure}[htb]
		\begin{center}
		\includegraphics[width=0.8\textwidth]{figures/piano1998tjibaou2.jpg}
		\caption{
		A completed building, within the complex.
		\citeauthor{piano1998tjibaou}: 
		\citetitle{piano1998tjibaou} 
		\cite{piano1998tjibaou}.
		Photo source: Wikimedia Commons.
		}		
		\label{fig:piano1998tjibaou2}
		\end{center}
		\end{figure}		
		
		\end{quote}
				

\clearpage
\citeSK{gehry2003laphil}
		\begin{quote}
		\small
		Designed by Frank Gehry and completed in 2003, this is the new home for
		Los Angeles Philharmonic, with seating capacity of 2,265.
		
		\begin{figure}[htb]
		\begin{center}
		\includegraphics[width=0.99\textwidth]{figures/gehry2003laphil1.jpg}
		\caption{
		Variations of the auditorium model (on the back wall).
		From the exhibition ``Frank O. Gehry since 1997'', Vitra Design Museum, Weil am Rhein, Germany.
		\citeauthor{gehry2003laphil}: 
		\citetitle{gehry2003laphil} 
		\cite{gehry2003laphil}.
		Photo adapted from source: 
		\url{http://ca.phaidon.com/agenda/architecture/picture-galleries/2011/january/25/the-work-of-frank-gehry-from-1997-to-today/?idx=3}. 		
		}		
		\label{fig:gehry2003laphil}
		\end{center}
		\end{figure}		

		\begin{figure}[htb]
		\begin{center}
		\includegraphics[width=0.99\textwidth]{figures/gehry2003laphil2.jpg}
		\caption{
		Completed building.
		\citeauthor{gehry2003laphil}: 
		\citetitle{gehry2003laphil} 
		\cite{gehry2003laphil}.
		Photo source: Wikimedia Commons. 
		}		
		\label{fig:gehry2003laphil2}
		\end{center}
		\end{figure}		
				
		\end{quote}


		
\clearpage
\citeSK{calatrava2004turningtorso}
		\begin{quote}
		\small
		The construction of this residential tower was completed in 2004 in Malm\"{o}, Sweden.
		The architect, Santiago Calatrava, made a number of preliminary sketches
		with alternative configurations, and appearances.
		\begin{figure}[htb]
		\begin{center}
		\includegraphics[width=0.95\textwidth]{figures/calatrava2004turningtorso1.jpg}
		\caption{
		Multiple sketches made by Calatrava for the Turning Torso building.		
% 		\citeauthor{calatrava2004turningtorso}: 
% 		\citetitle{calatrava2004turningtorso} 
% 		\cite{calatrava2004turningtorso}.
		Material adapted from 
		\citeauthor{jodidio2009calatrava}: 
		\citetitle{jodidio2009calatrava} 
		\cite{jodidio2009calatrava}.}		
		\label{fig:calatrava2004turningtorso1}
		\end{center}
		\end{figure}
		
		\begin{figure}[htb]
		\begin{center}
		\includegraphics[width=0.7\textwidth]{figures/calatrava2004turningtorso2.jpg}
		\caption{
		Completed building.
		\citeauthor{calatrava2004turningtorso}: 
		\citetitle{calatrava2004turningtorso} 
		\cite{calatrava2004turningtorso}.
		Photo source: Wikimedia Commons. 
		}		
		\label{fig:calatrava2004turningtorso2}
		\end{center}
		\end{figure}		

		\end{quote}
				
		
\clearpage
\citeSK{carusostjohn2005brickhouse}
		\begin{quote}
		\small
		A family house completed in 2005 in London.
		\begin{figure}[htb]
		\begin{center}
		\includegraphics[width=0.95\textwidth]{figures/carusostjohn2005brickhouse1.jpg}
		\caption{
		Several models (in this case, ten of them) were created before a choice on the final conceptual design has been made.
		\citeauthor{carusostjohn2005brickhouse}: 
		\citetitle{carusostjohn2005brickhouse} 
		\cite{carusostjohn2005brickhouse}.
		Photo source: 
		\citeauthor{bell2010thenewmodernhouse}: 
		\citetitle{bell2010thenewmodernhouse} 
		\cite{bell2010thenewmodernhouse}. 
		}		
		\label{fig:carusostjohn2005brickhouse1}
		\end{center}
		\end{figure}
				
		\begin{figure}[htb]
		\begin{center}
		\includegraphics[width=0.7\textwidth]{figures/carusostjohn2005brickhouse2.jpg}
		\caption{
		Completed house, interior view.
		\citeauthor{carusostjohn2005brickhouse}: 
		\citetitle{carusostjohn2005brickhouse} 
		\cite{carusostjohn2005brickhouse}.
		Photo source: \url{http://www.carusostjohn.com/projects/brick-house/}.
		}		
		\label{fig:carusostjohn2005brickhouse2}
		\end{center}
		\end{figure}
		
		\end{quote}
		

\clearpage
\citeSK{calatrava2005chicagospire}
		\begin{quote}
		\small
		A residential tower design (not built) by Santiago Calatrava.
		\begin{figure}[htb]
		\begin{center}
		\includegraphics[width=0.90\textwidth]{figures/calatrava2005chicagospire1.jpg}
		\caption{
		A number of sketches were made by Calatrava before a choice on the final design has been decided on.
		\citeauthor{calatrava2005chicagospire}: 
		\citetitle{calatrava2005chicagospire} 
		\cite{calatrava2005chicagospire}.
		Sketches adapted from: 
		\citeauthor{jodidio2009calatrava}: 
		\citetitle{jodidio2009calatrava} 
		\cite{jodidio2009calatrava}. 
		}		
		\label{fig:jodidio2009calatrava1}
		\end{center}
		\end{figure}
				
		\begin{figure}[htb]
		\begin{center}
		\includegraphics[width=0.99\textwidth]{figures/calatrava2005chicagospire2.jpg}
		\caption{
		Tower rendering, view.
		\citeauthor{calatrava2005chicagospire}: 
		\citetitle{calatrava2005chicagospire} 
		\cite{calatrava2005chicagospire}.
		Rendering adapted from: Wikipedia.
		}		
		\label{fig:calatrava2005chicagospire2}
		\end{center}
		\end{figure}
		
		\end{quote}
		
				

\clearpage
\citeSK{broekx2006lenaerts}
		\begin{quote}
		\small
		This is a detached single-family residence project completed in 2006.		
		\begin{figure}[htb]
		\begin{center}
		\includegraphics[width=0.60\textwidth]{figures/broekx2006lenaerts1.jpg}
		\caption{
		Top: preliminary multiple sketches, done on a single sheet of paper.
		Bottom: a progression of models showing evolution of the massing.
		\citeauthor{broekx2006lenaerts}: 
		\citetitle{broekx2006lenaerts} 
		\cite{broekx2006lenaerts}.
		Photo source: 
		\citeauthor{carusostjohn2005brickhouse}: 
		\citetitle{carusostjohn2005brickhouse} 
		\cite{bell2010thenewmodernhouse}. 
		}		
		\label{fig:broekx2006lenaerts1}
		\end{center}
		\end{figure}
				
		\begin{figure}[htb]
		\begin{center}
		\includegraphics[width=0.99\textwidth]{figures/broekx2006lenaerts2.jpg}
		\caption{
		Completed house.
		\citeauthor{broekx2006lenaerts}: 
		\citetitle{broekx2006lenaerts} 
		\cite{broekx2006lenaerts}.
		Photo source: \url{http://www.vai.be/en/project/lenaerts-thijs-house}.
		}		
		\label{fig:broekx2006lenaerts2}
		\end{center}
		\end{figure}
		
		\end{quote}
		

				
% \clearpage
% \citeSK{foster2006valerygergiev}
% 		\begin{quote}
% 		\small
% 		This is a design proposal (Fig. \ref{fig:foster2006valerygergiev}) 
% 		by Martha Tsigkari (Foster+Partners) for a cultural centre to be built in
% 		Vladikavkaz, North Ossetia. Four separate buildings (music school, library and 
% 		two concert venues) that are variations of one single conceptual idea (i.e. river pebbles). 
% 		These four different (but related) building designs were produced using one single 
% 		parametric CAD model.
% 		
% 		\begin{figure}[htb]
% 		\begin{center}
% 		\includegraphics[width=0.8\textwidth]{figures/foster2006valerygergiev1.jpg}
% 		%\includegraphics[width=0.45\textwidth]{figures/foster2006valerygergiev2.jpg}
% 		\caption{Martha Tsigkari's Valery Gergiev Cultural Centre.
% 		This is a complex 
% 		consisting of four buildings, % (music school, library, two concert halls), 
% 		all variations of a single conceptual idea.
% 		\citeauthor{foster2006valerygergiev}: 
% 		\citetitle{foster2006valerygergiev} 
% 		\cite{foster2006valerygergiev}.
% 		Photo source: Wikimedia Commons. 
% 		}		
% 		\label{fig:foster2006valerygergiev}
% 		\end{center}
% 		\end{figure}
% 				
% 		\begin{figure}[htb]
% 		\begin{center}
% 		\includegraphics[width=0.8\textwidth]{figures/foster2006valerygergiev2.jpg}
% 		\caption{
% 		Completed complex (rendering).
% 		\citeauthor{foster2006valerygergiev}: 
% 		\citetitle{foster2006valerygergiev} 
% 		\cite{foster2006valerygergiev}.
% 		}		
% 		\label{fig:foster2006valerygergiev2}
% 		\end{center}
% 		\end{figure}
% 		
% 		\end{quote}
		
		
\clearpage
\citeSK{gehry2007iac}
		\begin{quote}
		\small
		Headquarters for the InterActiveCorp, Inc., located in Manhattan, New York City, and completed in 2007.
		%, this office building gives an impression of a two-storey building, but in reality has ten stories.
		
		\begin{figure}[htb]
		\begin{center}
		\includegraphics[width=0.99\textwidth]{figures/gehry2007iac1.jpg}
		\caption{
		Variations of the building model.
		From the exhibition ``Frank O. Gehry since 1997'', Vitra Design Museum, Weil am Rhein, Germany. 
		\citeauthor{gehry2007iac}: 
		\citetitle{gehry2007iac} 
		\cite{gehry2007iac}.
		Photo source: \url{http://ca.phaidon.com/agenda/architecture/picture-galleries/2011/january/25/the-work-of-frank-gehry-from-1997-to-today/?idx=4}.		
		}		
		\label{fig:gehry2007iac}
		\end{center}
		\end{figure}		
		
		\begin{figure}[htb]
		\begin{center}
		\includegraphics[width=0.99\textwidth]{figures/gehry2007iac2.jpg}
		\caption{
		Completed building.
		\citeauthor{gehry2007iac}: 
		\citetitle{gehry2007iac} 
		\cite{gehry2007iac}.
		Photo source: Wikimedia Commons.
		}		
		\label{fig:gehry2007iac2}
		\end{center}
		\end{figure}		
		
		\end{quote}

\clearpage
\citeSK{maleki2008reinterpreting}
		\begin{quote}
		\small
		Ancient Persian domes which have delicate Islamic drawings projected onto them. 
		\begin{figure}[htb]
		\begin{center}
		\includegraphics[width=0.6\textwidth]{figures/maleki2008reinterpreting.jpg}
		\caption{
		Several variations of Rasmi domes.
		\citeauthor{maleki2008reinterpreting}: 
		\citetitle{maleki2008reinterpreting} 
		\cite{maleki2008reinterpreting}.}		
		\label{fig:maleki2008reinterpreting}
		\end{center}
		\end{figure}		
		\end{quote}
					
			
\clearpage
\citeSK{gehry2011sprucestreet}
		\begin{quote}
		\small
		Also known as Beekman Tower, or 8 Spruce Street tower, or commercially as {\em New York by Gehry},
		this mixed-use tower based in Lower Manhattan was completed in 2011. 
		\begin{figure}[htb]
		\begin{center}
		\includegraphics[width=0.99\textwidth]{figures/gehry2011sprucestreet1.jpg}
		\caption{
		Variations of the tower model.
		From the exhibition ``Frank O. Gehry since 1997'', Vitra Design Museum, Weil am Rhein, Germany.
		\citeauthor{gehry2011sprucestreet}: 
		\citetitle{gehry2011sprucestreet} 
		\cite{gehry2011sprucestreet}.
		Photo adapted from source: 
		\url{http://ca.phaidon.com/agenda/architecture/picture-galleries/2011/january/25/the-work-of-frank-gehry-from-1997-to-today/?idx=3}.
		}		
		\label{fig:gehry2011sprucestreet}
		\end{center}
		\end{figure}		
		
		\begin{figure}[htb]
		\begin{center}
		\includegraphics[width=0.4\textwidth]{figures/gehry2011sprucestreet2.jpg}
		\caption{
		Completed building.
		\citeauthor{gehry2011sprucestreet}: 
		\citetitle{gehry2011sprucestreet} 
		\cite{gehry2011sprucestreet}.
		Photo source: Wikimedia Commons.
		}		
		\label{fig:gehry2011sprucestreet2}
		\end{center}
		\end{figure}		
		
		\end{quote}

		

						
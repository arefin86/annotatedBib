	
%%%%%%%%%%%%%%%%%%%%%%%%%%%%%%%%%%%%%%%%%%%%%%%%%%%%%%%%%%%%%%%%%%%%%%%%%%%%%%%%%%%%%%%%%%%%%%%%%%%%
%%%%%%%%%%%%%%%%%%%%%%%%%%%%%%%%%%%%%%%%%%%%%%%%%%%%%%%%%%%%%%%%%%%%%%%%%%%%%%%%%%%%%%%%%%%%%%%%%%%%
%%%%%%%%%%%%%%%%%%%%%%%%%%%%%%%%%%%%%%%%%%%%%%%%%%%%%%%%%%%%%%%%%%%%%%%%%%%%%%%%%%%%%%%%%%%%%%%%%%%%
%%%%%%%%%%%%%%%%%%%%%%%%%%%%%%%%%%%%%%%%%%%%%%%%%%%%%%%%%%%%%%%%%%%%%%%%%%%%%%%%%%%%%%%%%%%%%%%%%%%%
%%%%%%%%%%%%%%%%%%%%%%%%%%%%%%%%%%%%%%%%%%%%%%%%%%%%%%%%%%%%%%%%%%%%%%%%%%%%%%%%%%%%%%%%%%%%%%%%%%%%
%%%%%%%%%%%%%%%%%%%%%%%%%%%%%%%%%%%%%%%%%%%%%%%%%%%%%%%%%%%%%%%%%%%%%%%%%%%%%%%%%%%%%%%%%%%%%%%%%%%%
\newpage
\section{Introduction}

In order to better understand the design problem at hand---its nature, its structure, and 
its limits---designers usually generate a number of {\em alternative design solutions} 
(in further text, {\em alternatives}) to the same problem.

% As designers work on a given design problem,
% they commonly generate a number of {\em alternative design solutions} 
% (in further text, {\em alternatives}) to the same problem. 
% %Why do they do that? Mostly, 
% They do this in order  %probe about, feel around, sense, and ultimately 
% to better understand the very 
% design problem at hand---its nature, its structure, and its limits---before they embark
% on refining the designs further.

%\medskip

However, contemporary computer-aided design (CAD) tools lack appropriate support for 
working with %generating, managing, visualizing and evaluating 
design alternatives (and sets thereof).
%
Due to this fact, an entire dimension of how designers actually work 
is largely missing from such tools, thus narrowing their scope of effective usage.
%
This annotated bibliography represents a first step in my quest to explore 
how to add effective and efficient support for working with sets of alternatives to CAD tools.
It is organized as follows:

\begin{itemize}

\item
In the {\em Breadth} section, I list references which give broad support
to solving the problem of how to build better computational tools for working with 
sets of design alternatives. The areas included are: 
design and cognition, 
computer-aided design,
modeling and representations,
creativity and problem solving,
% design requirements and goal engineering,
% prototype-based computing,
\& histories and versioning.

\item
In the {\em Depth} section, I focus on references which are strongly related to:
1) computer-assisted generation, management,
visualization, and evaluation of multiple alternatives, and 
2) real-life examples of alternatives ``in the wild'', that is, references
which show how designers (in the most general sense) use sets of design alternative in practice.
%thus helping establish the relevance of this research.


\item Finally, in the {\em Research Methods} section, 
I %outline my research plan, and 
list a number of references relevant
to how to evaluate information systems, and conduct research in general. % (with heavy emphasis on evaluating interactivity), 
This includes references on  
qualitative, quantitative and mixed methods,
%ways to study one's own design activity,
pitfalls to be avoided,
and ways to measure a system's quality of creativity support.

\end{itemize}

\begin{flushright}
\em
Sini\v{s}a Kolari\'c\\
%Burnaby, BC\\
Spring 2012
\end{flushright}
% Design problems aren't well defined and thus have no clearly defined strategy as of to how to solve them.
% 
% I start with this very general and encompassing definition of what it means to design:
% 
% \begin{quote}
% Everyone designs who devises courses of action aimed at changing existing situations into preferred ones.
% --- Herbert Simon
% \end{quote}
% 
